\documentclass[twoside]{report}

% Package Import
\usepackage{amsmath, amsfonts, amssymb, amsthm}
\usepackage[paperwidth=5.5in,paperheight=7.5in,margin=.5in]{geometry}
\usepackage{graphicx}
\usepackage{color}
\usepackage{enumerate}
\usepackage{mathrsfs}
\usepackage{mdframed}
  \mdfdefinestyle{prf}{
    topline=false,
    bottomline=false,
    rightline=false,
    innertopmargin=1ex,
    skipabove=0pt}
\usepackage{fancyhdr}
  \setlength{\headheight}{.5in}
  \setlength{\headsep}{2em}
  \pagestyle{fancy}

% Shortcut Macros
\newcommand{\done}{$\blacksquare$}
\newcommand{\hruleskip}{\vspace{1em}\hrule\vspace{2em}}
\newcommand{\br}{\centering{ * \quad * \quad *} \\  }
\newcommand{\Ob}{\textrm{Ob}}
\newcommand{\Hom}{\textrm{Hom}}
\newcommand{\C}{\mathbb{C}}
\newcommand{\R}{\mathbb{R}}
\newcommand{\Z}{\mathbb{Z}}
\newcommand{\N}{\mathbb{N}}
\newcommand{\fanC}{\mathscr{C}}
\newcommand{\fanD}{\mathscr{D}}
\newcommand{\cleanbr}{\vspace{1em}\noindent}

% Lecture Headers
\newcounter{Lecture}
\newcommand{\newLec}[1]{
  \stepcounter{Lecture}
  \noindent{\Large\bf Lecture \arabic{Lecture}} \, #1 \hfill  \rule[1ex]{2.5in}{.1pt} \vspace{1em}
}

% Theorem Stuff
\newtheoremstyle{myts}
  {1pt} % Space above
  {1em} % Space below
  {} % Body font
  {0pt} % Indent size
  {\bf} % Head font
  {} % headpunc
  { } % headspace
  {\thmname{#1}\thmnumber{#2}} % headspec
\theoremstyle{myts}
\newcounter{c}[Lecture]
\newtheorem{dfn}[c]{Definition \arabic{Lecture}.}
\newtheorem{thm}[c]{Theorem \arabic{Lecture}.}
\newtheorem{lma}[c]{Lemma \arabic{Lecture}.}
\newtheorem{cor}[c]{Corollary \arabic{Lecture}.}
\newtheorem*{epl}{Example}
\newtheorem*{nte}{Note}
\newcounter{ex}[Lecture]
\newtheorem{exc}[ex]{Exercise }

% Proof Environment
\newenvironment{prf}{
  \noindent\begin{mdframed}[style=prf]}{\end{mdframed} \vspace{1em}
}

\begin{document}

\begin{titlepage}
  \centering
  {\it Lecture Notes by Jonathan Alcaraz (UCR)} \\
  
  \vfill
  
  {\Huge Complex Analysis} \\
  \vspace{1em}
  {Math 210A} \\
  {Fall 2017} \\

  \vfill
  
  {Based on Lectures by} \\
  \vspace{1em}
  {\Large Dr. Wee Lang Gan} \\
  {\it University of California, Riverside}
  
  \vfill

  {\tiny last updated \today}
\end{titlepage}
\setcounter{page}{1}

\lhead[\thepage]{Alcaraz}
\chead[Complex]{Complex}
\rhead[Fall 2017 (Gan)]{\thepage}

\newLec{29 Sep 2017}

\cleanbr
{\large\scshape The Topology of the Complex Plane}

\cleanbr
\begin{dfn}
  Given $a\in\C$, $r>0$, define an \emph{open ball} by
  \[
    B(a,r) = \{ z\in\C : |z-a| < r \}
  \]
  and a \emph{closed ball} by
  \[
    \overline{B}(a,r) = \{ z\in\C : |z-a| \leq r \}
  \]
\end{dfn}

\begin{dfn}
  Take sets \( A\subseteq G\subseteq \C \). $A$ is said to be \emph{open in $G$} if for any \(a\in A\), there is some \(r>0\) such that \(B(a,r)\bigcap G\subseteq A\). $A$ is said to be \emph{closed in $G$} if \(G \backslash A\) is open in $G$.
\end{dfn}

\begin{dfn}
  A subset \(G \subseteq \C\) is said to be \emph{connected} if it has either of the following properties:
  \begin{itemize}
    \item If \( G = A\bigcup B\) where $A$, $B$ are open and disjoint, the \(A = \emptyset\) or \( B = \emptyset \).
    \item If \( A \subseteq G\) is both open in $G$ and closed in $G$, then \( A = \emptyset \) or \( A = G \).
  \end{itemize}
\end{dfn}

\begin{dfn}
  A \emph{segment} between complex numbers $z$ and $w$, denoted \( [z,w] \) is the set \( \{ tw + (1-t)z : t\in [0,1] \} \).
\end{dfn}

\begin{dfn}
  A \emph{polygon} from $a$ to $b$ is a set \( [a,z_1] \cup [z_1,z_2] \cup \cdots \cup [z_n,b] \).
\end{dfn}

\begin{thm}
  An open set $G$ is connected if and only if, for every \(a,b\in G\) there is a polygon from $a$ to $b$.
\end{thm}
\end{document}
