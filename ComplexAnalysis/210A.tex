\documentclass[twoside]{report}

% Package Import
\usepackage{amsmath, amsfonts, amssymb, amsthm}
\usepackage[paperwidth=5.5in,paperheight=7.5in,margin=.5in]{geometry}
\usepackage{graphicx}
\usepackage{color}
\usepackage{enumerate}
\usepackage{mathrsfs}
\usepackage{mdframed}
  \mdfdefinestyle{prf}{
    topline=false,
    bottomline=false,
    rightline=false,
    innertopmargin=1ex,
    skipabove=0pt}
\usepackage{fancyhdr}
  \setlength{\headheight}{.5in}
  \setlength{\headsep}{2em}
  \pagestyle{fancy}

% Shortcut Macros
\newcommand{\done}{$\blacksquare$}
\newcommand{\hruleskip}{\vspace{1em}\hrule\vspace{2em}}
\newcommand{\br}{\centering{ * \quad * \quad *} \\  }
\newcommand{\Ob}{\textrm{Ob}}
\newcommand{\Hom}{\textrm{Hom}}
\newcommand{\C}{\mathbb{C}}
\newcommand{\R}{\mathbb{R}}
\newcommand{\Z}{\mathbb{Z}}
\newcommand{\N}{\mathbb{N}}
\newcommand{\fanC}{\mathscr{C}}
\newcommand{\fanD}{\mathscr{D}}
\newcommand{\cleanbr}{\vspace{1em}\noindent}

% Lecture Headers
\newcounter{Lecture}
\newcommand{\newLec}[1]{
  \stepcounter{Lecture}
  \noindent{\Large\bf Lecture \arabic{Lecture}} \, #1 \hfill  \rule[1ex]{2.5in}{.1pt} \vspace{1em}
}

% Theorem Stuff
\newtheoremstyle{myts}
  {1pt} % Space above
  {1em} % Space below
  {} % Body font
  {0pt} % Indent size
  {\bf} % Head font
  {} % headpunc
  { } % headspace
  {\thmname{#1}\thmnumber{#2}} % headspec
\theoremstyle{myts}
\newcounter{c}[Lecture]
\newtheorem{dfn}[c]{Definition \arabic{Lecture}.}
\newtheorem{thm}[c]{Theorem \arabic{Lecture}.}
\newtheorem{lma}[c]{Lemma \arabic{Lecture}.}
\newtheorem{cor}[c]{Corollary \arabic{Lecture}.}
\newtheorem*{epl}{Example}
\newtheorem*{nte}{Note}
\newcounter{ex}[Lecture]
\newtheorem{exc}[ex]{Exercise }

% Proof Environment
\newenvironment{prf}{
  \noindent\begin{mdframed}[style=prf]}{\end{mdframed} \vspace{1em}
}

\begin{document}

\begin{titlepage}
  \centering
  {\it Lecture Notes by Jonathan Alcaraz (UCR)} \\
  
  \vfill
  
  {\Huge Complex Analysis} \\
  \vspace{1em}
  {Math 210A} \\
  {Fall 2017} \\

  \vfill
  
  {Based on Lectures by} \\
  \vspace{1em}
  {\Large Dr. Wee Lang Gan} \\
  {\it University of California, Riverside}
  
  \vfill

  {\tiny last updated \today}
\end{titlepage}
\setcounter{page}{1}

\lhead[\thepage]{Alcaraz}
\chead[Complex]{Complex}
\rhead[Fall 2017 (Gan)]{\thepage}

\newLec{29 Sep 2017}

\cleanbr
{\large\scshape The Topology of the Complex Plane}

\cleanbr
\begin{dfn}
  Given $a\in\C$, $r>0$, define an \emph{open ball} by
  \[
    B(a,r) = \{ z\in\C : |z-a| < r \}
  \]
  and a \emph{closed ball} by
  \[
    \overline{B}(a,r) = \{ z\in\C : |z-a| \leq r \}
  \]
\end{dfn}

\begin{dfn}
  Take sets \( A\subseteq G\subseteq \C \). $A$ is said to be \emph{open in $G$} if for any \(a\in A\), there is some \(r>0\) such that \(B(a,r)\bigcap G\subseteq A\). $A$ is said to be \emph{closed in $G$} if \(G \backslash A\) is open in $G$.
\end{dfn}

\begin{dfn}
  A subset \(G \subseteq \C\) is said to be \emph{connected} if it has either of the following properties:
  \begin{itemize}
    \item If \( G = A\bigcup B\) where $A$, $B$ are open and disjoint, the \(A = \emptyset\) or \( B = \emptyset \).
    \item If \( A \subseteq G\) is both open in $G$ and closed in $G$, then \( A = \emptyset \) or \( A = G \).
  \end{itemize}
\end{dfn}

\begin{dfn}
  A \emph{segment} between complex numbers $z$ and $w$, denoted \( [z,w] \) is the set \( \{ tw + (1-t)z : t\in [0,1] \} \).
\end{dfn}

\begin{dfn}
  A \emph{polygon} from $a$ to $b$ is a set \( [a,z_1] \cup [z_1,z_2] \cup \cdots \cup [z_n,b] \).
\end{dfn}

\begin{thm}
  An open set $G$ is connected if and only if, for every \(a,b\in G\) there is a polygon from $a$ to $b$.
\end{thm}

\begin{dfn}
  Given a subset \( A\subseteq \C\), we say \(z\in\C\) is a \emph{limit point} of $A$ if there exists a sequence \(\{a_n\}\) of distinct points in $A$ such that \( z = \lim_{n\to\infty} a_n \).
\end{dfn}

\begin{cor}
  A subset $A$ is closed if and only if $A$ contains all of its limit points.
\end{cor}

\begin{dfn}
  A subset \(A \subseteq\C\) is \emph{complete} if every Cauchy sequencce in $A$ converges in $A$.
\end{dfn}

\begin{cor}
  $A$ is complet if and only if $A$ is closed.
\end{cor}

\begin{dfn}
  A subset $A$ of $\C$ is \emph{compact} if every open cover of $A$ has a finite subcover. $A$ is \emph{sequentially compact} if every sequencec in $A$ has a subsequence which converges in $A$. 
\end{dfn}

\begin{dfn}
  A set \(A\subseteq\C\) is \emph{totally bounded} if for every \(\varepsilon > 0\) there exists \( a_1, \ldots, a_n \in A \) such that \( A\subseteq \bigcup_{i=1}^n B(a_i, \varepsilon)\). 
\end{dfn}

\begin{thm}
  The following are equivalent:
  \begin{enumerate}[(i)]
    \item $A$ is compact;
    \item Every infinie set in $A$ has limit point in $A$;
    \item $A$ is sequentially compact;
    \item $A$ is complete and totally bounded.
  \end{enumerate}
\end{thm}

\begin{cor}
  $A$ is compact if and only if $A$ is closed and bounded.
\end{cor}

\newLec{2 Oct 2017}

\begin{thm}
  Let \(A\subseteq\C\) and \( f : A \to \C \) be a continuous function. If $A$ is a compact (resp. connected), then \(f(A)\) is compact (resp. connected). Moreover, if $A$ is compact, then \(f(A)\)  is bounded and attains its bounds.
\end{thm}

\begin{dfn}
  A fuunction \( f: A\to \C\) is said to be \emph{uniformly continuous} if for every \(\varepsilon >0\), there is a \(\delta > 0 \) such that  \( |f(z) - f(w)| < \varepsilon | \) whenever \( | z-w | < \delta \).
\end{dfn}

\begin{nte}
  Uniform continuity implies standard pointwise continuity, but the converse need not be true. However, the converse is true when $A$ is compact.
\end{nte}

\begin{dfn}
  Let \( f: A\to\C \) and \( f_n: A\to\C \) be functions. We say the sequence \(\{f_n\}\) \emph{converges uniformly} to $f$ is for every \( \varepsilon > 0 \), there is some $N$ such that every \(n\geq N\) has the property \( |f_n(z) -f(z) | < \varepsilon \) for every \(z\in A\).
\end{dfn}

\begin{thm}
  If $f_n$ converges uniformly to $f$ on $A$, and $f_n$ are continuous, then $f$ is continuous.
\end{thm}

\begin{prf}
  Fix \( a\in A\). We want to show that $f$ is continuous at $a$. Let \(\varepsilon > 0 \). There is $n$ sufficiently large so that \( | f(z) - f_n(z) | < \frac{\varepsilon}{3} \) for any \( z\in \C \). Moreover, since $f_n$ continuous, there is a $\delta > 0$ such that \( | f_n(z) - f_n(a) | < \frac{\varepsilon}{3} \) whenever \( | z-a | < \delta \). So if \( |z-a| < \delta \), then
  \[
    |f(z) - f(a) | \leq |f(z) - f_n(z)| + |f_n(z) - f_n(a) | + | f_n(a) - f(a) | < \varepsilon
  \]
\end{prf}

\begin{dfn}
  Let \( u_n : A \to \C \) be a sequence of functions and define \( f = \sum_{n=1}^\infty u_n \) by
  \[
    f(z) = \lim_{N\to\infty} \sum_{n=1}^N u_n(z)
  \]
  We say \(\sum_{n=1}^\infty u_n \) \emph{converges uniformly} if the partial sums \( \sum_{n=1}^N u_n \)  converge uniformly in the aforementioned sense. 
\end{dfn}

\begin{thm}
  ({\scshape Weirstrass M-test}) Let \(A\subseteq\C\) and \( u_n : A \to \C \). Suppose there are numbers \( \{ M_n \} \) such that \(  | u_n(z) | \leq M_n \) for every \(z\in A\) and every $n$. If \( \sum_{n=1}^\infty M_n \), then \( \sum_{n=1}^\infty \) converges uniformly.
\end{thm}

\begin{prf}
  Define $f_N$ to be the $N$th partial sum of $u_n$. Since
  \begin{align*}
    | f_n(z) - f_m(z) | &\leq |u_{m+1}(z)| + \cdots + |u_{n}(z)| \\
      &\leq M_{m+1} + \cdots + M_{n}
  \end{align*}
  Since \(\sum M_n\) converges, the RHS can be made arbitrarily small for sufficiently large $m,n$, $f_n$ is pointwise Cauchy and hencce converges pointwise. Define $f$ to be the pointwise limit of $f_n$. So,
  \begin{align*}
    | f(z) - f_n(z) | &= | u_{n+1}(z) + u_{n+2} + \cdots | \\
      &\leq M_{n+1} + M_{n+2} + \cdots
  \end{align*}
  Again, since $M_n$ converges, this can be made arbitrarilly small. 
\end{prf}

\newLec{4 Oct 2017}

\cleanbr
{\scshape Power Series}

\begin{dfn}
  Let \(a_n \in \C \) We say the series \( \sum_{n=0}^\infty a_n \) \emph{converges} to \(z\in\C\) if for every \(\varepsilon > 0\), there is an $N$ such that for every \(m\geq N\), \( \left| \sum_{n=0}^m a_n - z \right| < \varepsilon \). We say the series \emph{converges absolutely} if \( \sum_{n=0}^\infty |a_n|\) converges. If a series does not converge, we say the series \emph{diverges}.
\end{dfn}

\begin{lma}
  Absolute converges implies convergence.
\end{lma}

\begin{prf}
  Define \(z_n\) to be the $n$-th partial sum of \(\{a_n\}\). If \(m>k\), then
  \begin{align*}
    | z_m - z_k | &= | a_{k+1} + \cdots + a_m | \\
      &\leq |a_{k+1}| + \cdots + |a_m|
  \end{align*}
  Since the series is absolutely convergent, then this value can be made arbitrarily small. So \(z_n\) is Cauchy and hence converges.
\end{prf}

\begin{dfn}
  A \emph{power series} about $a$ is a series of the form
  \[
   \sum_{n=0}^\infty a_n (z-a)^n
  \]
\end{dfn}


\begin{thm}
  For any power series \( \sum a_n (z-a)^n \), there is a nonnegative real number $R$ such that:
  \begin{enumerate}[(1)]
    \item \( |z-a| < R \) implies the series converges absolutely.
    \item \( |z-a| > R \) implies the series divverges.
  \end{enumerate}
  Moreover, if \(0<r<R\), then the series converges uniformly on \(\overline{B}(a,r)\).
  The unique $R$ satisfying these properties is called the \emph{radius of convergence} of the power series.
\end{thm}

\begin{prf}
  Recall the \emph{limit supremum} of a sequence defined by
  \[
    \limsup a_n = \lim_{n\to\infty} \left( \sup \{ a_m : m\geq n \} \right)
  \]
  We need only consider the case where \( a = 0 \). Define $R$ implicitly by
  \[
    \frac{1}{R} = \limsup |a_n|^{1/n}
  \]
  Suppose \( |z| < R \) and choose $r$ so that \( |z| < r < R \). Since \(\frac{1}{r} > \frac{1}{R} \), there exists $N$ such that for every \(n\geq N\), \( |a_n|^{\frac{1}{n}} < \frac{1}{r} \). So,
  \[
      \sum_{n=N}^\infty |a_n z^n| = \sum_{n=N}^\infty \left| (a_n^{\frac{1}{n}})^n \right| \, |z^n| < \sum_{n=N}^\infty \left( \frac{|z|}{r} \right)^n
  \]
  Since \( |z| < r \), this is a convergent geometric series. This proves (1).
  
  \noindent
  Suppose \( |z| > R \). Choose $r$ so that \(R<r<|z|\). Then \(\frac{1}{r} < \frac{1}{R} \), so there are infinitely many $n$ such that \( \frac{1}{r} < |a_n|^{\frac{1}{n}}\). Therefore the sequence \( |a_n z^n| > \left( \frac{|z|}{r} \right)^n \) diverges and hence the corresponding series diverges.

  \noindent
  Suppose \(0 < r < R\). Choose $\rho$ such that \( r < \rho < R \). So there exists $N$ such that for every \(n\geq N\), \( |z| ^{\frac{1}{n}} < \frac{1}{\rho} \). Therefore \(n\geq N\), \( |z| \leq r \) so \( |a_n z^n| \leq \left( \frac{r}{\rho} \right)^n \). So by the Weirstrass $M$-test, the series converges uniformly. 
\end{prf}

\begin{thm}
  If \( \lim \left| \frac{a_n}{a_{n+1}} \right| \) exists, then \( R = \lim \left| \frac{a_n}{a_{n+1}} \right| \)
\end{thm}

\begin{prf}
  Again, we may assume that \(a=0\). Let \( a = \lim \left| \frac{a_n}{a_{n+1}} \right| \). If \( |z| < r < a \), then there is some \(N\) such that for every \( n\geq N\), \( r < \left| \frac{a_n}{a_{n+1}} \right| \). Let \( B = \left| a_N \, r^N \right| \). Then
  \[
    B > \left| a_{N+1} \, r^{N+1} \right| > \cdots
  \]
  Hence, if \(n\geq N\), then 
  \[
    \left| a_n z^n \right| = \left| a_n z^n \right| \, \left| \frac{z^n}{r^n} \right| \leq B \left| \frac{z}{r} \right|^n
  \]
  
  \noindent
  Similarly, if \( |z| > r > a \), then there exists $N$ such that for every \( n\geq N \), \( \left| \frac{a_n}{a_{n+1}} \right| < r \). Let \(B = \left| a_N r^N \right| \). Then
  \[
    B = \left| a_N r^N \right| < \left| a_{N+1} \right| < \cdots
  \]
  So, if \(n\geq N\), then 
  \[
    \left| a_n z^n \right| = \left| a_n z^n \right| \, \left| \frac{z^n}{r^n} \right| \geq B \left| \frac{z}{r} \right|^n
  \]
\end{prf}

\begin{epl}
  Consider
  \[
    \sum_{n=0}^{\infty} \frac{z^n}{n!}
  \]
  The radius of convergence is \(\lim \left| \frac{n+1}{n} \right| = \infty \).

  \noindent
  Since this power series converges for any complex number, it defines a well-defined function called the \emph{exponential function} denoted:
  \[
    \exp (z) := e^z := \sum_{n=0}^{\infty} \frac{z^n}{n!}
  \]
\end{epl}

\end{document}
