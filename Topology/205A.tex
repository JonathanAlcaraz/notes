\documentclass[twoside]{report}

% Package Import
\usepackage{amsmath, amsfonts, amssymb, amsthm}
\usepackage[paperwidth=5.5in,paperheight=7.5in,margin=.5in]{geometry}
\usepackage{graphicx}
\usepackage{color}
\usepackage{enumerate}
\usepackage{mathrsfs}
\usepackage{mdframed}
  \mdfdefinestyle{prf}{
    topline=false,
    bottomline=false,
    rightline=false,
    innertopmargin=1ex,
    skipabove=0pt}
\usepackage{fancyhdr}
  \setlength{\headheight}{.5in}
  \setlength{\headsep}{2em}
  \pagestyle{fancy}

% Shortcut Macros
\newcommand{\done}{$\blacksquare$}
\newcommand{\hruleskip}{\vspace{1em}\hrule\vspace{2em}}
\newcommand{\br}{\centering{ * \quad * \quad *} \\  }
\newcommand{\Ob}{\textrm{Ob}}
\newcommand{\Hom}{\textrm{Hom}}
\newcommand{\C}{\mathbb{C}}
\newcommand{\R}{\mathbb{R}}
\newcommand{\Z}{\mathbb{Z}}
\newcommand{\N}{\mathbb{N}}
\newcommand{\fanC}{\mathscr{C}}
\newcommand{\fanD}{\mathscr{D}}
\newcommand{\cleanbr}{\vspace{1em}\noindent}

% Lecture Headers
\newcounter{Lecture}
\newcommand{\newLec}[1]{
  \stepcounter{Lecture}
  \noindent{\Large\bf Lecture \arabic{Lecture}} \, #1 \hfill  \rule[1ex]{2.5in}{.1pt} \vspace{1em}
}

% Theorem Stuff
\newtheoremstyle{myts}
  {1pt} % Space above
  {1em} % Space below
  {} % Body font
  {0pt} % Indent size
  {\bf} % Head font
  {} % headpunc
  { } % headspace
  {\thmname{#1}\thmnumber{#2}} % headspec
\theoremstyle{myts}
\newcounter{c}[Lecture]
\newtheorem{dfn}[c]{Definition \arabic{Lecture}.}
\newtheorem{thm}[c]{Theorem \arabic{Lecture}.}
\newtheorem{lma}[c]{Lemma \arabic{Lecture}.}
\newtheorem{cor}[c]{Corollary \arabic{Lecture}.}
\newtheorem*{epl}{Example}
\newtheorem*{nte}{Note}
\newcounter{ex}[Lecture]
\newtheorem{exc}[ex]{Exercise }

% Proof Environment
\newenvironment{prf}{
  \noindent\begin{mdframed}[style=prf]}{\end{mdframed} \vspace{1em}
}

\begin{document}

\begin{titlepage}
  \centering
  {\it Lecture Notes by Jonathan Alcaraz (UCR)} \\
  
  \vfill
  
  {\Huge Topology} \\
  \vspace{1em}
  {Math 205A} \\
  {Fall 2017} \\

  \vfill
  
  {Based on Lectures by} \\
  \vspace{1em}
  {\Large Dr. Frederick Willhelm} \\
  {\it University of California, Riverside}
  
  \vfill

  {\tiny last updated \today}
\end{titlepage}
\setcounter{page}{1}

\lhead[\thepage]{Alcaraz}
\chead[Topology]{Topology}
\rhead[Fall 2017 (Willhelm)]{\thepage}

\newLec{28 Sep 2017}

\cleanbr
{\large\scshape Introduction}

\cleanbr
This course is about the abstraction of continuity.

\cleanbr
A subset \(O\subseteq\R\) is said to be \emph{open} if for every point \(x\in O\), there is a value \(\varepsilon > 0\) such that \( B(x,\varepsilon) \subseteq O \).

\cleanbr
This idea of openness in Euclidean space can be used to define continuity of functions on Euclidean space.

\cleanbr
A function \(f:\R^n \to \R\) is \emph{continuous} if for any open \(O\subseteq\R\), \(f^{-1}(O)\) is open in $\R^n$.

\cleanbr
One could prove that this definition is equivalet to the standard $\varepsilon$-$\delta$ definition of continuity on Euclidean space.

\cleanbr
{\large\scshape Point-Set Topology}

\begin{dfn}
  Let $X$ be a set. A collection $\mathcal{T}$ of subsets of $X$ is called a \emph{topology} on $X$ if
  \begin{enumerate}[(i)]
    \item $\emptyset$ and $X$ are in $\mathcal{T}$;
    \item Any union of elements of $\mathcal{T}$ is in $\mathcal{T}$;
    \item The intersection of finitely many elements of $\mathcal{T}$ is in $\mathcal{T}$.
  \end{enumerate}
  A set together with a topology $\mathcal{T}$ is called a \emph{topological space} denoted \( (X,\mathcal{T}) \) (or just $X$ if the topology is understood). Elements of $\mathcal{T}$ are said to be \emph{open sets in $X$}.
\end{dfn}

\begin{epl}
  \cleanbr
  \begin{itemize}
    \item The \emph{discrete topology} on a set is simply the collection of all subsets.
    \item The \emph{indisccrete topology} on a set $X$ is the most trivial topolgy, \(\{\emptyset, X\}\).
  \end{itemize}
\end{epl}

\begin{dfn}
  If \(\mathcal{T}_1\) and \(\mathcal{T}_2\) are topologies on the same set with the property \( \mathcal{T}_1 \subseteq \mathcal{T}_2 \), then \(\mathcal{T}_2\) is said to be \emph{finer} than \(\mathcal{T}_1\). If neither \( \mathcal{T}_1 \subseteq \mathcal{T}_2 \) nor \( \mathcal{T}_2 \subseteq \mathcal{T}_1 \), these topologies are said to be \emph{incomparable}.
\end{dfn}

\begin{dfn}
  A function \( f : (X,\mathcal{T}) \to (Y,\mathcal{T}') \) is called \emph{continuous} if \(f^{-1}(V) \in \mathcal{T}\) for every \(V\in\mathcal{T}'\), that is, the preimage of open sets are open. A \emph{homeomorphism} is a bijective continuous funcction whose inverse is continuous.
\end{dfn}

\cleanbr
{\bf The Fundamental Question of Topology}  Given two topological spaces, determine if they are homeomorphic.

\end{document}
