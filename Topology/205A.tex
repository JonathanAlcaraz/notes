\documentclass[twoside]{report}

% Package Import
\usepackage{amsmath, amsfonts, amssymb, amsthm}
\usepackage[paperwidth=5.5in,paperheight=7.5in,margin=.5in]{geometry}
\usepackage{graphicx}
\usepackage{color}
\usepackage{enumerate}
\usepackage{mathrsfs}
\usepackage{mdframed}
  \mdfdefinestyle{prf}{
    topline=false,
    bottomline=false,
    rightline=false,
    innertopmargin=1ex,
    skipabove=0pt}
\usepackage{fancyhdr}
  \setlength{\headheight}{.5in}
  \setlength{\headsep}{2em}
  \pagestyle{fancy}

% Shortcut Macros
\newcommand{\done}{$\blacksquare$}
\newcommand{\hruleskip}{\vspace{1em}\hrule\vspace{2em}}
\newcommand{\br}{\centering{ * \quad * \quad *} \\  }
\newcommand{\Ob}{\textrm{Ob}}
\newcommand{\Hom}{\textrm{Hom}}
\newcommand{\C}{\mathbb{C}}
\newcommand{\R}{\mathbb{R}}
\newcommand{\Z}{\mathbb{Z}}
\newcommand{\N}{\mathbb{N}}
\newcommand{\fanC}{\mathscr{C}}
\newcommand{\fanD}{\mathscr{D}}
\newcommand{\cleanbr}{\vspace{1em}\noindent}

% Lecture Headers
\newcounter{Lecture}
\newcommand{\newLec}[1]{
  \stepcounter{Lecture}
  \noindent{\Large\bf Lecture \arabic{Lecture}} \, #1 \hfill  \rule[1ex]{2.5in}{.1pt} \vspace{1em}
}

% Theorem Stuff
\newtheoremstyle{myts}
  {1pt} % Space above
  {1em} % Space below
  {} % Body font
  {0pt} % Indent size
  {\bf} % Head font
  {} % headpunc
  { } % headspace
  {\thmname{#1}\thmnumber{#2}} % headspec
\theoremstyle{myts}
\newcounter{c}[Lecture]
\newtheorem{dfn}[c]{Definition \arabic{Lecture}.}
\newtheorem{thm}[c]{Theorem \arabic{Lecture}.}
\newtheorem{lma}[c]{Lemma \arabic{Lecture}.}
\newtheorem{cor}[c]{Corollary \arabic{Lecture}.}
\newtheorem*{epl}{Example}
\newtheorem*{nte}{Note}
\newcounter{ex}[Lecture]
\newtheorem{exc}[ex]{Exercise }

% Proof Environment
\newenvironment{prf}{
  \noindent\begin{mdframed}[style=prf]}{\end{mdframed} \vspace{1em}
}

\begin{document}

\begin{titlepage}
  \centering
  {\it Lecture Notes by Jonathan Alcaraz (UCR)} \\
  
  \vfill
  
  {\Huge Topology} \\
  \vspace{1em}
  {Math 205A} \\
  {Fall 2017} \\

  \vfill
  
  {Based on Lectures by} \\
  \vspace{1em}
  {\Large Dr. Frederick Willhelm} \\
  {\it University of California, Riverside}
  
  \vfill

  {\tiny last updated \today}
\end{titlepage}
\setcounter{page}{1}

\lhead[\thepage]{Alcaraz}
\chead[Topology]{Topology}
\rhead[Fall 2017 (Willhelm)]{\thepage}

\newLec{28 Sep 2017}

\cleanbr
{\large\scshape Introduction}

\cleanbr
This course is about the abstraction of continuity.

\cleanbr
A subset \(O\subseteq\R\) is said to be \emph{open} if for every point \(x\in O\), there is a value \(\varepsilon > 0\) such that \( B(x,\varepsilon) \subseteq O \).

\cleanbr
This idea of openness in Euclidean space can be used to define continuity of functions on Euclidean space.

\cleanbr
A function \(f:\R^n \to \R\) is \emph{continuous} if for any open \(O\subseteq\R\), \(f^{-1}(O)\) is open in $\R^n$.

\cleanbr
One could prove that this definition is equivalet to the standard $\varepsilon$-$\delta$ definition of continuity on Euclidean space.

\cleanbr
{\large\scshape Point-Set Topology}

\begin{dfn}
  Let $X$ be a set. A collection $\mathcal{T}$ of subsets of $X$ is called a \emph{topology} on $X$ if
  \begin{enumerate}[(i)]
    \item $\emptyset$ and $X$ are in $\mathcal{T}$;
    \item Any union of elements of $\mathcal{T}$ is in $\mathcal{T}$;
    \item The intersection of finitely many elements of $\mathcal{T}$ is in $\mathcal{T}$.
  \end{enumerate}
  A set together with a topology $\mathcal{T}$ is called a \emph{topological space} denoted \( (X,\mathcal{T}) \) (or just $X$ if the topology is understood). Elements of $\mathcal{T}$ are said to be \emph{open sets in $X$}.
\end{dfn}

\begin{epl}
  \cleanbr
  \begin{itemize}
    \item The \emph{discrete topology} on a set is simply the collection of all subsets.
    \item The \emph{indisccrete topology} on a set $X$ is the most trivial topolgy, \(\{\emptyset, X\}\).
  \end{itemize}
\end{epl}

\begin{dfn}
  If \(\mathcal{T}_1\) and \(\mathcal{T}_2\) are topologies on the same set with the property \( \mathcal{T}_1 \subseteq \mathcal{T}_2 \), then \(\mathcal{T}_2\) is said to be \emph{finer} than \(\mathcal{T}_1\). If neither \( \mathcal{T}_1 \subseteq \mathcal{T}_2 \) nor \( \mathcal{T}_2 \subseteq \mathcal{T}_1 \), these topologies are said to be \emph{incomparable}.
\end{dfn}

\begin{dfn}
  A function \( f : (X,\mathcal{T}) \to (Y,\mathcal{T}') \) is called \emph{continuous} if \(f^{-1}(V) \in \mathcal{T}\) for every \(V\in\mathcal{T}'\), that is, the preimage of open sets are open. A \emph{homeomorphism} is a bijective continuous funcction whose inverse is continuous.
\end{dfn}

\noindent
{\bf The Fundamental Question of Topology}  Given two topological spaces, determine if they are homeomorphic.
\cleanbr

\newLec{3 October 2017}

\begin{dfn}
  Let $X$ be a set. A \emph{basis} for a topology on $X$ is a collection $\mathcal{B}$ of subsets of $X$ such that
  \begin{enumerate}[(i)]
    \item For every \(x\in X\), there is a \(B\in\mathcal{B}\) such that \(x\in B\)
    \item If \( B_1,B_2 \in\mathcal{B} \) and \( x\in B_1\bigcap B_2 \), then there is some \(B_3\in\mathcal{B}\) such that \(x\in B_3\subseteq B_1\bigcap B_2 \).
  \end{enumerate}
\end{dfn}

\begin{dfn}
  Given a basis $\mathcal{B}$ on a set $X$, the \emph{topology generated by $\mathcal{B}$} is defined by: $U$ is open if and only if for every \(x\in U\), there is some \(B\in\mathcal{B}\) such that \(x\in B\subseteq U\).
\end{dfn}

\begin{nte}
  It is left as an exercise to confirm that sets with this property do indeed form a topology on $X$.
\end{nte}

\begin{epl}
\noindent
  \begin{enumerate}[(i)]
    \item The open intervals of $\R$ generate the standard topology on $\R$.
    \item Open balls in $\R^n$.
  \end{enumerate}
\end{epl}

\begin{lma}
  The topology generated by a basis $\mathcal{B}$ is precisely the collection of unions of sets in $\mathcal{B}$.
\end{lma}

\begin{lma}
  The topology \(\mathcal{T}'\) is finer than the topology \(\mathcal{T}\) if and only if for every \(x\in X\) and \(B\in\mathcal{B}\) with \(x\in B\), there is a set \(B'\in\mathcal{B}'\) such that \(x\in B'\subseteq B\).
\end{lma}

\begin{lma}
  Suppose \(\mathcal{C}\) is a collection of open sets of $X$ such that for every \(x\in X\) and neighborhood $U$ of $x$, there is a set \(C\in\mathcal{C}\) such that \(x\in C\subseteq U\). Then $\mathcal{C}$ is a basis of $X$ that generates the same topology of $X$.
\end{lma}

\begin{dfn}
  A collection $\mathcal{S}$ is a \emph{subbasis} for a topology if \(X = \bigcup_{S\in \mathcal{S}} S\).
\end{dfn}

\br

\begin{dfn}
  The \emph{product topology} on the cartesian product of topology spaces \(X\times Y\) has a basis \(\{U\times V : U\subseteq X, V\subseteq Y \textrm{ open respectively} \}\).
\end{dfn}

\begin{nte}
  One could check this indeed generates a topology on \(X\times Y\). In fact, the finite intersections of elements of this collection are again in this collection.
\end{nte}

\begin{thm}
  If $\mathcal{B}$ and $\mathcal{C}$ are bases for $X$ and $Y$ respecively, then
  \[
    \mathcal{D} = \{ B\times C : B\in\mathcal{B}, C\in\mathcal{C} \}
  \]
  is a basis for $X\times Y$ with the product topology.
\end{thm}

\begin{prf}
  Let \( W\subseteq X\times Y\) open and \( (x,y) \in W \). By the definition of the product topology, there is a product of open sets in $X$ and $Y$ respectively such that \( (x,y) \in U\times V \subseteq W \). Since \(x\in U\) and $U$ is open, there is a set \(B\in\mathcal{B}\) such that \( x \in B\subseteq U\). Similarly, there is a set \(C\in\mathcal{C}\) such that \(y\in C\subseteq V\). So
  \[
    (x,y) \in B\times C \subseteq U\times V \subseteq W
  \]
  and \(B\times C\in \mathcal{D}\).
\end{prf}

\begin{dfn}
  The maps \( \pi_1 : X\times Y \to X : (x,y) \mapsto x \) and \( \pi_2: X\times Y \to Y : (x,y) \mapsto y \) are called \emph{projection maps} and are continuous since \( \pi_1^{-1} (U) = U \times Y \).
\end{dfn}

\br

\begin{dfn}
  Let \( (X, \mathcal{T} ) \) be a topological space and $Y$ a subset of $X$, then the collection
  \[
    \mathcal{T}_Y = \{ Y\cap U : U\in\mathcal{T} \}
  \]
  is the \emph{subspace topology} on $Y$.
\end{dfn}

\begin{lma}
  The subspace topology is indeed a topology on $Y$.
\end{lma}

\begin{prf}
  The axioms we desire follow from basic facts of set theory:
  \begin{enumerate}[(i)]
    \item \( \emptyset = Y \cap \emptyset \)
    \item \( Y = Y\cap X \)
    \item \( \bigcap_{n=1}^N (U_n \cap Y) = \left(\bigcap_{n=1}^N U_n \right) \cap Y \)
    \item \( \bigcup_{\alpha\in J} ( U_\alpha \cap Y ) = \left( \bigcup U_\alpha \right) \cap Y \)
  \end{enumerate}
\end{prf}

\begin{lma}
  If $\mathcal{B}$ is a basis for the topology $X$, then \( \mathcal{B}_Y := \{ B\cap Y : B\in\mathcal{B} \}\) is a basis for \(\mathcal{T}_Y\).
\end{lma}

\begin{prf}
  Let \( U\cap Y\) be open in $Y$ with $U$ open in $X$ and \(y\in U\cap Y\). Then there is a set \(B\in\mathcal{B}\) such that \(y\subseteq B \subseteq U \). So \( B\cap Y \in \mathcal{B}_Y \) and \(y \in B\cap Y \subseteq U\cap Y \).
\end{prf}

\begin{nte}
  Since we have this notion of the subspace topology, the concept of openness is a relative property.
\end{nte}

\end{document}
