\documentclass[twoside]{report}

% Package Import
\usepackage{amsmath, amsfonts, amssymb, amsthm}
\usepackage[paperwidth=5.5in,paperheight=7.5in,margin=.5in]{geometry}
\usepackage{graphicx}
\usepackage{color}
\usepackage{enumerate}
\usepackage{mathrsfs}
\usepackage{mdframed}
  \mdfdefinestyle{prf}{
    topline=false,
    bottomline=false,
    rightline=false,
    innertopmargin=1ex,
    skipabove=0pt}
\usepackage{fancyhdr}
  \setlength{\headheight}{.5in}
  \setlength{\headsep}{2em}
  \pagestyle{fancy}

% Shortcut Macros
\newcommand{\done}{$\blacksquare$}
\newcommand{\hruleskip}{\vspace{1em}\hrule\vspace{2em}}
\newcommand{\br}{\centering{ * \quad * \quad *} \\  }
\newcommand{\Ob}{\textrm{Ob}}
\newcommand{\Hom}{\textrm{Hom}}
\newcommand{\C}{\mathbb{C}}
\newcommand{\R}{\mathbb{R}}
\newcommand{\Z}{\mathbb{Z}}
\newcommand{\N}{\mathbb{N}}
\newcommand{\fanC}{\mathscr{C}}
\newcommand{\fanD}{\mathscr{D}}
\newcommand{\cleanbr}{\vspace{1ex}\noindent}

% Lecture Headers
\newcounter{Lecture}
\newcommand{\newLec}[1]{
  \stepcounter{Lecture}
  \noindent{\Large\bf Lecture \arabic{Lecture}} \, #1 \hfill  \rule[1ex]{2.5in}{.1pt} \vspace{1em}
}

% Theorem Stuff
\newtheoremstyle{myts}
  {1pt} % Space above
  {1em} % Space below
  {} % Body font
  {0pt} % Indent size
  {\bf} % Head font
  {} % headpunc
  { } % headspace
  {\thmname{#1}\thmnumber{#2}} % headspec
\theoremstyle{myts}
\newcounter{c}[Lecture]
\newtheorem{dfn}[c]{Definition \arabic{Lecture}.}
\newtheorem{thm}[c]{Theorem \arabic{Lecture}.}
\newtheorem{lma}[c]{Lemma \arabic{Lecture}.}
\newtheorem{cor}[c]{Corollary \arabic{Lecture}.}
\newtheorem*{epl}{Example}
\newtheorem*{nte}{Note}
\newcounter{ex}[Lecture]
\newtheorem{exc}[ex]{Exercise }

% Proof Environment
\newenvironment{prf}{
  \noindent\begin{mdframed}[style=prf]}{\end{mdframed} \vspace{1em}
}

\begin{document}

\begin{titlepage}
  \centering
  {\it Lecture Notes by Jonathan Alcaraz (UCR)} \\
  
  \vfill
  
  {\Huge Geometry} \\
  \vspace{1em}
  {Math 232A} \\
  {Fall 2017} \\

  \vfill
  
  {Based on Lectures by} \\
  \vspace{1em}
  {\Large Dr. Frederick Willhelm} \\
  {\it University of California, Riverside}
  
  \vfill
\end{titlepage}
\setcounter{page}{1}

\lhead[\thepage]{Alcaraz}
\chead[Geometry]{Geometry}
\rhead[Fall 2017 (Willhelm)]{\thepage}

\newLec{28 Sep 2017}

\noindent
{\large\scshape Introduction}

\cleanbr
  Given a smooth surface \( S \subseteq \R^3 \), we'd like to compare and contrast with \( \R^n\). In doing so, we may ask for analogs to lines, angles, volume, distance, length, et cetera.

\cleanbr
In fact, we can even consider these analogs in more abstract smooth manifolds. For example:
\begin{itemize}
  \item \(\R P^n\)
  \item the Torus
  \item the Klein Bottle
\end{itemize}

\cleanbr
While these space can be embedded into $\R^n$, we don't usually consider them this way, but rather thin of them abstractly.

\cleanbr
{\large\scshape Riemannian Manifolds}

\begin{dfn}
  A \emph{Riemannian Manifold} \( (M, g)\) is a \(C^\infty\)-manifold together with a smoothly varying inner product $g$ on $M$.
\end{dfn}

\begin{dfn}
  An \emph{Inner Product} is a map \( \langle \cdot , \cdot \rangle : V \times V \to \R \) that is bilinear, symmetric, and positive definite.
\end{dfn}

\begin{nte}
  Really there are inifinitely many inner products in a Riemmanian Manifold since each point has its own tangent space. We sometimes say $g_p$ is the inner product at $p$. We say that $g$ is a \emph{Riemannian metric}, though this shouldn't be confused with standard notions of a metric.
\end{nte}

\end{document}
